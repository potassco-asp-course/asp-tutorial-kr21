%------------------------------------------------------------
\lecture{Solving}{solving}
%------------------------------------------------------------
\part{Solving}
% ----------------------------------------------------------------------
\begin{frame}{Reasoning modes}
\vfill
\begin{center}{%
\begin{picture}(300,120)(-150,-60)
\put(-80,+40){\makebox(0,0){\framebox(80,20){Problem}}}
\put(-80,-40){\makebox(0,0){\framebox(80,20){Logic Program}}}
\put(+80,+40){\makebox(0,0){\framebox(80,20){Solution}}}
\put(+80,-40){\makebox(0,0){\framebox(80,20){\alert{\textbf{Stable Models}}}}}
\put(-80,+30){\vector(0,-1){60}}
\put(-40,-40){\vector(+1,0){80}}
\put(+80,-30){\vector(0,+1){60}}
\put(-110,  0){\makebox(0,0){{Modeling}}}
\put(+120,  0){\makebox(0,0){{Interpreting}}}
\put(   0,-55){\makebox(0,0){{Solving}}}
\end{picture}}
\end{center}
\end{frame}
% ----------------------------------------------------------------------
\input{introduction/reasoning-modes}
% ----------------------------------------------------------------------
\section{Conflict-driven constraint learning}
%
\input{solving/cdcl-motivation}
\input{solving/dpll}
\input{solving/cdcl}
% ----------------------------------------------------------------------
\section{Engine}
%
\input{systems/architecture}
% ----------------------------------------------------------------------

%%% Local Variables:
%%% mode: latex
%%% TeX-master: "tutorial"
%%% End:
